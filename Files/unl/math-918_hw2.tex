\documentclass[11pt]{amsart}
\usepackage[utf8]{inputenc}
\usepackage[margin=1in, headheight=14.5pt]{geometry}
\usepackage{math-hw}
\pagestyle{fancy}
\fancyhead[L]{Math 918}
%\fancyhead[C]{MATH 665}
\fancyhead[R]{HW 2}

\usepackage[margin=1in, headheight=14.5pt]{geometry}
\usepackage{parskip}

%%%%%%%%%%%%%%%%%%%%%%%%%%%%%%%%%%
% fonts
\usepackage{calligra}
\usepackage{mathrsfs}
\usepackage[mathscr]{euscript}
\usepackage{bbm}

\usepackage{enumitem}
\usepackage{hyperref}
%\usepackage[colorlinks,  citecolor=green, linkcolor = black, urlcolor = cyan]{hyperref} % hyperlinks for cross-referencing; colorlinks is recommended because otherwise the boxes obscure the roman numerals; linkcolor is changed to blue after the table of contents (see firstpages.tex). As of 2021, red is not allowed by Rackham.
\usepackage[nameinlink]{cleveref}
%\usepackage{backref}
\usepackage{url} % allow hyperlink in reference
\usepackage[style=alphabetic,url=false,backref=true]{biblatex}
\addbibresource{zotero-export.bib}
\setlength{\droptitle}{-4em}


\newtoggle{solution}
\toggletrue{solution}


\iftoggle{solution}{
\includeversion{solnenv}
\includeversion{commenv}
}{
\excludeversion{solnenv}
\excludeversion{commenv}
}





\title{Homework 2}
%\author{Anna Brosowsky}
\date{\vspace{-4em}}

\begin{document}


\maketitle\thispagestyle{fancy}

Remember you are allowed to discuss with classmates (or an AI tool), but that you need to tell me who/what you discussed with + the final submitted writeup should be your own work.

\begin{enumerate}[itemsep=3em]
\item Let $R$ be a ring and let $f_1,\ldots, f_t$ be a regular sequence. 
\begin{enumerate}
\item Prove the following lemma:

\begin{lemma}[Colon Capturing]
For all $0\leq i<t$, we have
\[\langle f_1,\ldots, f_{i}\rangle : \langle f_{i+1}\rangle = \langle f_1,\ldots, f_i\rangle,\]
where when $i=0$ the left side of the colon is the ideal generated by NO elements, i.e., the zero ideal.
\end{lemma}

\item Using the above lemma and other facts about regular sequences, prove the following proposition:

\begin{prop}
Assume that $R$ has char $p>0$. Let $I=\langle f_1,\ldots, f_t\rangle$. Then 
\[
I^{[p^e]}:I = I^{[p^e]} + \langle (f_1\cdots f_t)^{p^e-1}\rangle.
\]
\end{prop}

\end{enumerate}

\item  \begin{enumerate}

\item Let $S$ be an $F$-finite regular local ring, let $\{F_*b_i\}_{i=1}^t$ be basis for $F_*S$, and let $\Phi\in \hom_S(F_*S,S)$ be the generating map. Suppose that $g\in S$ can be written as $g=\sum_i a_i^pb_i$ for some $a_i\in S$. Show that
\[
\Phi(F_*(gS)) = \langle a_1,\ldots, a_t\rangle.
\]



\item The \emph{Frobenius root} of an ideal $I$ (written $I^{[1/p]}$) is the smallest ideal $J$ such that $I\subset J^{[p]}$. Prove that in the setup of part (a), we have
\[
\langle g\rangle ^{[1/p]} = \langle a_1,\ldots, a_t\rangle.
\]


\end{enumerate}

\item Let $S=k[x_1,\ldots, x_n]$ and let $f$ be a homogeneous polynomial of degree $d$. Using Fedder's criterion\footnote{Recall that Fedder also applies for homogeneous ideals in a (standard) graded ring, taking $\mf m =$ the homogeneous maximal ideal} prove that if $d>n$, then $S/f$ is not $F$-split.




\item Let $k$ be a char $p>0$ field and let $R=k[[x,y,z]]/\langle x^2+y^3+z^7\rangle$. Prove that $R$ is never $F$-split in any characteristic.


\item Let $X$ be a $4\times 4$ matrix of indeterminates, and let $S=\bb Z/3[X]$ be the polynomial ring over these indeterminates (so, a 16 variable polynomial ring). Let $I_2$ be the ideal of $2\times 2$ minors. Using Macaulay2 and Fedder's criterion, verify that $S/I_2$ is $F$-split. \footnote{For me on the HCC, this check took about 1 hour, and used about 5gb of memory. If you want to double-check your code for errors, first try the much-faster $p=2$ version (which is also $F$-split!)}\vspace{1ex}\\
\emph{Allowed Functions:} The only thing from \verb|TestIdeals| you are allowed to use is the \verb|frobenius| function (so, no isFPure!). However, you are free to use ANY other functions available in base Macaulay2 or any other packages.\footnote{Explore the documentation... you might find something useful, like \texttt{minors}, or how to quickly make a ring with 16 variables!} \vspace{1ex} \\
\emph{Submission:} Turn in your M2 code, HCC .submit file (if you used one), AND the output to your code. Format flexible: can be a screenshot if you did it in interactive mode, or copy-pasting the file contents in a verbatim environment into tex, or whatever works best for you.


\end{enumerate}


\end{document}
