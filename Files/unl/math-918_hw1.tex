\documentclass[11pt]{amsart}
\usepackage[utf8]{inputenc}
\usepackage[margin=1in, headheight=14.5pt]{geometry}
\usepackage{math-hw}
\pagestyle{fancy}
\fancyhead[L]{Math 918}
\fancyhead[R]{HW 1}

\usepackage[margin=1in, headheight=14.5pt]{geometry}

%%%%%%%%%%%%%%%%%%%%%%%%%%%%%%%%%%
% fonts
\usepackage{calligra}
\usepackage{mathrsfs}
\usepackage[mathscr]{euscript}

\usepackage{enumitem}
\usepackage{hyperref}
%\usepackage[colorlinks,  citecolor=green, linkcolor = black, urlcolor = cyan]{hyperref} % hyperlinks for cross-referencing; colorlinks is recommended because otherwise the boxes obscure the roman numerals; linkcolor is changed to blue after the table of contents (see firstpages.tex). As of 2021, red is not allowed by Rackham.
\usepackage[nameinlink]{cleveref}
%\usepackage{backref}
\setlength{\droptitle}{-4em}


\newtoggle{solution}
\toggletrue{solution}
%\togglefalse{solution}


\iftoggle{solution}{
\includeversion{solnenv}
\includeversion{commenv}
}{
\excludeversion{solnenv}
\excludeversion{commenv}
}





\title{Homework 1}
%\author{Anna Brosowsky}
\date{\vspace{-4em}}

\begin{document}


\maketitle\thispagestyle{fancy}

Remember you are allowed to discuss with classmates (or an AI tool), but that you need to tell me who/what you discussed with + the final submitted writeup should be your own work.

If you solve a problem using a computer algebra system (e.g., Macualay2 or Singular) that is allowed, just provide some code instead of just saying ``by Macaulay2''! Also please do so in the ``spirit'' of the problem, i.e., don't just use the function \verb|isFPure| :)

\begin{enumerate}[itemsep=4em]
\item Prove the following statement:
\begin{prop}
Let $(R,\mf m,k)$ be a local ring. The number of minimal generators for $F_*^eR$ as an $R$-module is
\[
[k:k^{p^e}]\cdot \dim_k(R/\mathfrak m^{[p^e]})
\]
\end{prop}

\item (Exercise 5 in Ma--Polstra) Prove that if $R$ is essentially of finite type\footnote{essentially of finite type = a localization of something of finite type} over an $F$-finite field, then $R$ is also $F$-finite. Do this via first proving each of the following three facts:
\begin{enumerate}
\item If $R$ is $F$-finite, then $R/I$ is $F$-finite for all ideals $I$.


\item If $R$ is $F$-finite, then $W^{-1}R$ is $F$-finite for all multiplicative sets $W$.


\item If $R$ is $F$-finite, then $R[x]$ and $R[[x]]$ are $F$-finite for an indeterminate $x$.



\end{enumerate}

\item (Exercise 1 in Ma--Polstra) Prove that if there exists an $e_0>0$ such that $F_*^{e_0}R$ is a finite $R$-module, then in fact $F_*^eR$ is a finite $R$-module for all $e>0$.


\item Let $R=\bb F_2[x^2,x^3]$. Prove that $R$ is \emph{not} $F$-split.\footnote{Hint: You are welcome to use the generators \& relations for $F_*R$ we found in class, no need to reprove!}\footnote{Note: This problem is definitely doable by hand using what we've learned as of 1/22, but you are also welcome to use any other methods for testing $F$-splitting that we learn in-class between now and when this HW is due.}


\end{enumerate}


\end{document}
